\documentclass[a4paper,12pt]{report}
%\usepackage[left=1.5cm, top=1cm, right=1cm, bottom=20mm]{geometry}

%%% Работа с русским языком
\usepackage{cmap}					% поиск в PDF
\usepackage{mathtext} 				% русские буквы в формулах
\usepackage[T2A]{fontenc}			% кодировка
\usepackage[utf8]{inputenc}			% кодировка исходного текста
\usepackage[english,russian]{babel}	% ло кализация и переносы
%%\usepackage{textcomp}

%%% Дополнительная работа с математикой
\usepackage{amsmath,amsfonts,amssymb,amsthm,mathtools} 
%\usepackage{textcomp}

%% Номера формул
\usepackage{xymtexpdf}
\usepackage{chmst-pdf}

%%% Работа с картинками
\usepackage{graphicx}  % Для вставки рисунков
\usepackage{rotating}
\setlength\fboxsep{3pt} % Отступ рамки \fbox{} от рисунка
\setlength\fboxrule{1pt} % Толщина линий рамки \fbox{}
\usepackage{wrapfig} % Обтекание рисунков текстом

%%% Работа с таблицами
\usepackage{array,tabularx,tabulary,booktabs} % Дополнительная работа с таблицами
\usepackage{longtable}  % Длинные таблицы
\usepackage{multirow} % Слияние строк в таблице


%%% Программирование

%%%tikz
\usepackage{tikz}
\usetikzlibrary{calc}
\usetikzlibrary{patterns}
\usetikzlibrary{shadows}
\usepackage{indentfirst} % Красная строка


\usepackage[os=win, mackeys=symbols]{menukeys}
\usepackage{enumitem}

\usepackage[russian,noabbrev]{cleveref}
\usepackage{hyperref}

%%% Заголовок
\author{Ivan Anashkin}
\title{Методическое указание}
\date{\today}

\begin{document}

\chapter{Заявка в РФФИ}
\url{ http://www.rfbr.ru/rffi/ru/contest/o_2043343} -- ссылка на условия конкурса

\section{Основные данные проекта}

\textbf{Название проекта:} Молекулярно-статистическое моделирование процесса ректификации

\textbf{Ключевые слова:} ректификация, метод Монте-Карло, межмолекулярное взаимодействие

\textbf{Аннотация, публикуемая на сайте Фонда (не более 0,5 стр., в том числе кратко – актуальность, уровень значимости и научная новизна исследования; ожидаемые результаты и их значимость; аннотация будет опубликована на сайте Фонда, если Проект получит поддержку):} 
Проект нацелен на разработку метода молекулярно-статистического моделирования процесса ректификации и соответствующего программного обеспечения. Разрабатываемый подход основывается на сочетании метода теоретических тарелок и ансамбля Гиббса для расчета процесса ректификации. Основными исходными данными для моделирования являются молекулярная структура и межмолекулярное взаимодействие разделяемых веществ. Разрабатываемое программное обеспечение позволит рассчитывать распределение температуры, давления и состава смеси по тарелкам (или высоте колонны). 

\textbf{Название проекта (на английском языке):} Molecular-statistical simulation of the rectification process

\textbf{Ключевые слова (на английском языке)} rectification, Monte Carlo method, intermolecular interaction


\textbf{Аннотация Проекта на английском язык (объемом не более 0,5 стр.; в том числе кратко – актуальность, уровень фундаментальности и научная новизна; ожидаемые результаты и их значимость)}
The project is aimed at developing a method and software for molecular-statistical simulation of the rectification . The approach being developed is based on a combination of the theoretical plates method  and the Gibbs ensemble to calculate the rectification. The basic initial data for simulation are the molecular structure and the intermolecular interaction of the substances to be separated. The developed software will allow to calculate the distribution of temperature, pressure and composition of the mixture on the distillation plates (or the height of the column).


\section{Участники проекта}

\section{Организация}

\section{Содержание}
\textbf{Описание фундаментальной научной задачи, на решение которой направлено исследование:}
Данный проект направлен на решение проблемы моделирования химико-технологических систем при условии наличия информации лишь о химической структуре веществ, участвующих в процессе.

\textbf{Актуальность исследования:}
В настоящее время методы молекулярно~- статистического моделирования (метод молекулярной динамики и метод Монте-Карло) широко применяются при исследованиях во многих областях наук. Так, в химической технологии данные методы могут применятся при расчете термодинамических, кинетических, равновесных свойств веществ и их смесей; моделировании процессов адсорбции на поверхности, массопереноса через мембрану и т.д. В настоящее время применимость данных методов ограничена лишь вычислительной производительностью современных компьютеров.
В данной работе будет решаться задача молекулярно-статистического моделирования процесса ректификации. Фундаментальный подход при рассмотрении химической технологии предполагает наименьшее использование эмпирических методов. Одной из основных идей является замена описания фазового равновесия эмпирическим и полуэмпирическими моделями молекулярно-статистическими методами. Свойства веществ, участвующих в процессе, диктуются лишь их межмолекулярным взаимодействием и могут быть рассчитаны с использованием квантовохимических методов (данной тематике посвящены работы [Jäger B. et al. Ab initio virial equation of state for argon using a new nonadditive three-body potential // J. Chem. Phys. 2011. Vol. 135, № 8. P. 084308.;	Vogel E. et al. Ab initio pair potential energy curve for the argon atom pair and thermophysical properties for the dilute argon gas. II. Thermophysical properties for low-density argon // Mol. Phys. 2010. Vol. 108, № 24. P. 3335–3352.
;	Jäger B. et al. State-of-the-art ab initio potential energy curve for the krypton atom pair and thermophysical properties of dilute krypton gas // J. Chem. Phys. 2016. Vol. 144, № 11. P. 114304.
;	Bich E., Hellmann R., Vogel E. Ab initio potential energy curve for the neon atom pair and thermophysical properties for the dilute neon gas. II. Thermophysical properties for low-density neon // Mol. Phys. 2008. Vol. 106, № 6. P. 813–825.
;	Bock S., Bich E., Vogel E. A new intermolecular potential energy surface for carbon dioxide from ab initio calculations // Chem. Phys. 2000. Vol. 257, № 2–3. P. 147–156.]). Описанный подход, основанный лишь на  информации о химической структуре участвующих в процессе веществ, является актуальным в связи с тем, что каждый день синтезируется множество новых веществ и проведение экспериментальных исследований свойств для каждого из них не представляется возможным.

\textbf{Анализ современного состояния исследований в данной области (приводится обзор исследований в данной области со ссылками на публикации в научной литературе):}
Молекулярно~- статистические методы получили широкое распространение в научных исследованиях в связи с развитием вычислительной техники. Основы молекулярного моделирования и алгоритмы программ изложены в книге [Allen M.P., Tildesley D.J. Computer simulation of liquids. Oxford: Clarendon Press, 1989. 385 p.]. Так молекулярно~- статистические методы используются для расчета множества физико-химических свойств: фазового равновесия [ Panagio\-topoulos A.Z. Direct determination of phase coexistence properties of fluids by Monte Carlo simulation in a new ensemble // Mol. Phys. 1987. Vol. 61, № 4. P. 813–826.], коэффициентов диффузии [Guevara-Carrion G., Vrabec J., Hasse H. Prediction of self-diffusion coefficient and shear viscosity of water and its binary mixtures with methanol and ethanol by molecular simulation // J. Chem. Phys. 2011. Vol. 134, № 7. P. 074508.], коэффициентов вязкости [Hess B. Determining the shear viscosity of model liquids from molecular dynamics simulations // J. Chem. Phys. 2002. Vol. 116, № 1. P. 209.] и т.д. Стоит отметить развитие методов молекулярного моделирования процессов, например разработано несколько подходов для моделирования мембранных процессов [Heffelfinger G.S., Swol F. van. Diffusion in Lennard-Jones fluids using dual control volume grand canonical molecular dynamics simulation (DCV-GCMD) // J. Chem. Phys. 1994. Vol. 100, № 10. P. 7548.].
Анализ литературных источников показал, что аналогов описываемого в данной работе метода не предлагалось. В существующих методах расчета ректификационных колонн молекулярно-статистические методы используются для расчета физико-химических свойств. При этом рассчитанные значения аппроксимируются эмпирическими или полуэмпирическими формулами.

\textbf{Цель и задачи Проекта:}
Целью проекта является разработка молекулярно~- статистического метода расчета процесса ректификации. Для достижения поставленной цели будут решаться следующие задачи:
\begin{itemize}
	\item разработка метода молекулярного моделирования процесса ректификации на основе: метода теоретических тарелок, NVE ансабля и ансамбля Гиббса;
	\item разработка алгоритма разработанного метода;
	\item программная реализация алгоритма с использованием программно-аппаратной платформы CUDA;
	\item валидация результатов расчетов фазового равновесия программы по литературным данным;
	\item расчет ректификации на бинарной леннард-джонсовской смеси с использованием разработанной программы.
\end{itemize}

\textbf{Научная новизна исследования, заявленного в Проекте (формулируется новая научная идея, обосновывается новизна предлагаемой постановки и решения заявленной проблемы):}
В отличие от других методов расчета ректификации, предлагаемый подход не использует каких-либо эмпирических и полуэмпирических аппроксимаций свойств участвующих в процессе веществ. Согласно предлагаемому методу ректификационная колонна делится на участки, аналогичные теоретическим тарелкам (теоретическим ступеням), свойства на каждой из тарелок будут рассчитываться молекулярно-статистическими методами. Термодинамические условия на каждой ступени будут рассчитываться из закона сохранения энергии. Предлагаемый подход эффективен в случае возникновения проблем при расчете фазового равновесия: отсутствие экспериментальных данных о фазовом равновесии; невозможность описания фазового равновесия через бинарные коэффициенты (в случае сложного взаимодействие компонентов смеси между собой), большая вычислительная эффективность в случае многокомпонентных смесей. 

Разрабатываемый метод обладает следующими достоинствами: учитывает, что мольный поток газа и жидкости может быть различным по высоте колонны, возможность введения дополнительных массовых и тепловых потоков в колонну, возможность моделирования нестационарных процессов. 
 
\textbf{Предлагаемые подходы и методы, и их обоснование для реализации цели и задачи исследований (Развернутое описание предлагаемого исследования; форма изложения должна дать возможность эксперту оценить новизну идеи Проекта, соответствие подходов и методов исследования поставленным целям и задачам, надежность получаемых результатов):}
В основе предлагаемой модели лежит комбинация метода Монте-Карло в ансамбле Гиббса и метода теоретических тарелок.

В основе ансамбля Гиббса лежит исследование фазового пространства двух объемов в которых содержаться молекулы рассматриваемых веществ. Над системами проводятся некоторые манипуляции над молекулами: их перемещение, вращение, изменение объема самой системы, перенос молекул из одного объема в другой.  При этом каждый раз по приращению энергии всей системы высчитывается вероятность того что это изменение системы будет принято.

В методе теоретических тарелок колонна разделяется на тарелки, на каждой из которых достигаются условия фазового равновесия, таким образом покидающие тарелку фазы находятся в равновесии. Далее после вычисления количества теоретических тарелок, требуемых для разделения, необходимо соотнести их с количеством реальных тарелок через эффективность тарелки по Мэрфри (в случае тарельчатых колонн) или определить высоту слоя насадки, соответствующую одной тарелке (в случае колонн с непрерывным контактом фаз).

При расчете процесса ректификации по методу теоретических тарелок необходима информация о фазовом равновесии смеси веществ. В настоящее время для этого используются модели для описания коэффициентов активности (Вильсона, NRTL и другие), однако данные методы для определения параметров модели требуют экспериментальные данные. В предлагаемом подходе для расчета фазового равновесия используется лишь данные о межмолекулярном взаимодействии веществ. Силы межмолекулярного взаимодействия могут быть получены с использованием квантовохимических методов. 

В основе идеи лежит разделение колонны на теоретические тарелки. На каждой из тарелок  будет рассчитываться фазовое равновесие с использованием ансамбля Гиббса. После достижения условий фазового равновесия на каждой из тарелок, объемы смещаются друг относительно друга. Объем с газовой фазой идет на верхнюю тарелку, с жидкой фазой на нижнюю. В отличие от метода теоретических тарелок, условия на тарелках после перемещения объемов будут рассчитываться не по условиям фазового равновесия, а на основании закона сохранения энергии. Так будет подсчитываться энергия газовой и жидкой фаз, пришедших на теоретическую тарелку. С использованием NVE ансамбля [Lustig R. Microcanonical Monte Carlo simulation of thermodynamic properties // J. Chem. Phys. 1998. Vol. 109, № 20. P. 8816–8828.] будет вычисляться температура на тарелке, при этом энергия будет подсчитываться как сумма энергий газовой и жидкой фаз. Полученное значение температуры будут использоваться при расчете фазового равновесия на следующей итерации.

Суммарный объем колонны рассчитывается как сумма объемов каждой из ячеек. При этом потоки исходной смеси, кубового остатка и дистиллята можно представит в безразмерном виде, относительно объема колонны. Дистиллят и кубовый остаток отводятся в виде определенной доли молекул с верхней и нижней тарелок.

В связи с тем, что в расчете будет использоваться баланс энергии, можно вводить дополнительные потоки тепла на каждой из тарелок в виде энергии. Так можно вводить подогрев куба, или потери энергии на каждой из тарелок.

Планируется также анализ перехода системы к стационарному состоянию: изменение состава, количества молекул на тарелке в каждой из фаз, температур и других свойств на тарелке от времени. Метод Монте-Карло используется только для расчета равновесных состояний и не позволяет без каких либо модификаций рассматривать процессы проходящие во времени. В данной работе можно рассматривать нестационарный процесс, введя время за которое устанавливается равновесие на тарелке. Одним из решений данной проблемы является сопоставление расчета с экспериментальными данными. Поэтому в данной работе предлагается рассматривать изменения системы относительно циклов перемещения фаз между тарелками.

\textbf{Ожидаемые результаты научного исследования и их научная и прикладная значимость:}
В результате выполнения проекта будет разработан метод моделирования процесса ректификации и создано программное обеспечение реализующее данный метод. Разработанные программы позволят рассчитывать температуры, давления и состав на тарелках в зависимости от потока, состава исходной смеси, мощности кубового подогревателя, количества тарелок и других параметров. 

Несмотря на то, что в настоящее время предлагаемый метод сложно использовать при проектировании химико-технологических процессов в связи с высокой вычислительной сложностью, развитие вычислительной техники в будущем открывает перспективу его использования в будущем. Так предлагаемый метод может быть полезен при отсутствии экспериментальных данных о фазовом равновесии. В отличие от других методов расчета данный метод не предусматривает прямого использования информации о фазовом равновесии смеси (моделей для описания коэффициентов активности). А основывается на использовании потенциалов межмолекулярного взаимодействия, которые могут быть рассчитаны с использованием методов квантовой химии. 

\textbf{Общий план работ на весь срок реализации Проекта (форма представления информации должна дать возможность эксперту оценить реализуемость заявленного плана работы и риски его невыполнения; общий план реализации Проекта даётся с разбивкой по годам):}
Общий план выполнения проекта можно разделить на несколько последовательных этапов:
\begin{itemize}
	\item (2018 г., январь - март) разработка и описание молекулярно~- статистического метода моделирования процесса ректификации;
	\item (2018 г. апрель - 2019 г. февраль) программная реализация алгоритмов. Для повышения требуемой производительности будет использована программно-аппаратная технология CUDA. Разработанное программное обеспечение будет выложено под открытой лицензией на сайте  github.com;
	\item (2019 г. март - июнь) тестирование приложений, сравнение результатов по расчету фазового равновесия;
	\item (2019 г. июль - сентябрь) моделирование ректификации бинарной леннард-джонсовской смеси и публикация статей по результатам моделирования. 
\end{itemize} 

\textbf{План работ на первый год реализации Проекта (план предоставляется с учетом содержания работ каждого из участников Проекта, предполагаемых поездок):}
В первый год выполнения проекта планируется:
\begin{itemize}
	\item разработка метода моделирования процесса ректификации молекулярно статистическими методами;
	\item разработка алгоритмов и программная реализация разработанной модели;
	\item отладка и тестирование приложений, проверка правильности расчета фазового равновесия на чистых леннард-джонсовских флюидах и смесях.
\end{itemize}

Анашкин (руководитель проекта): разработка молекулярно~- статистического метода и алгоритмов для расчета процесса ректификации, определение используемых для разработки средств, постановка задач моделирования фазового равновесия и анализ результатов.

Казанцев (исполнитель): программирование с использованием программно~- аппаратной технологии CUDA. 


\textbf{Ожидаемые научные результаты за первый год реализации Проекта (форма изложения должна дать возможность провести экспертизу результатов и оценить возможную степень выполнения заявленного в Проекте плана работы):}
В первый год выполнения проекта будет разработан метод моделирования процесса ректификации и описаны используемые алгоритмы. Частично будет реализована программная реализация используемых алгоритмов, в частности: блок ввода исходных данных для моделирования, блок задания начального распределения молекул по теоретическим тарелкам, блок изменения конфигурации на тарелках.


\textbf{Имеющийся у коллектива научный задел по Проекту (указываются полученные результаты, разработанные программы и методы, экспериментальное оборудование, материалы и информационные ресурсы, имеющиеся в распоряжении коллектива для реализации Проекта):}

Члены коллектива имеют опыт разработки программ молекулярного моделирования (исходный код некоторых программ выложен в открытый доступ https://github.com/kstu). Для проведения моделирования в распоряжении научной группы есть компьютерный класс, оснащенный видеокартами поддерживающими технологию CUDA. 

\textbf{Публикации (не более 15) участников коллектива, включая Руководителя проекта, наиболее близко относящиеся к Проекту за последние 5 лет (для каждой публикации при наличии указать ссылку в сети Интернет для доступа эксперта к аннотации или полному тексту публикации):}
\begin{enumerate}
	\item Anashkin I., Klinov A. Thermodynamic behavior of charged Lennard-Jones fluids // J. Mol. Liq. 2017. Vol. 234. P. 424–429. doi:10.1016/j.mol\-liq.2017.03.113.
	\item	Anashkin I.P., Klinov A.V. Determining the parameters of the potential of intermolecular interaction by the Zeno line // Russ. J. Phys. Chem. A. 2013. Vol. 87, № 11. P. 1781–1788.doi: 10.1134/S0036024413110034
	\item	Анашкин И.П., Клинов А.В. Молекулярно-статистическое моделирование процесса первапорации // Вестник Казанского Технологического Университета. 2013. Vol. 16, № 19. P. 7–13.
	\item	Клинов А.В., Анашкин И.П., Фазлыев А.Р. Описание процесса первапорации леннард-джонсовской смеси на основе метода молекулярной динамики и модели «растворения диффузии». // ММТТ-28 Ярославль, 2015.
	\item	Анашкин И.П. Массоперенос через мембрану из гибридного оксида кремния в процессах первапорационного разделения жидких смесей: Диссертация на соискание ученой степени кандидата технических наук. Казань: КНИТУ, 2014.
	\item	Анашкин И.П., Клинов А.В. Фазовое равновесие флюидов с межмолекулярным потенциалом Карра-Коновалова // Вестник Казанского Технологического Университета. 2012. Vol. 15, № 2. P. 7–9.
	\item	Анашкин И.П., Клинов А.В. Расчет фазового равновесия этана и этилена с использованием новой модели межмолекулярного взаимодействия // Вестник Казанского Технологического Университета. 2012. Vol. 15, № 11. P. 84–85.
	\item	Клинов А.В., Анашкин И.П., Новосёлова Ю.В., Якупова Т.Р. Фазовые диаграммы ионных жидкостей на основе грубой модели межмолекулярного взаимодействия // Вестник Казанского Технологического Университета. 2015. Vol. 18, № 14. P. 190–192.
\end{enumerate}
	


\chapter{Содержание}

\section*{Введение}

Молекулярно-статистические методы находят широкое применение в научных исследованиях. С развитием вычислительной техники сложность моделируемых систем возрастала. Усложнялись описание потенциального взаимодействия молекул, что позволило увеличить согласование расчета с экспериментом. Также увеличивалось количество моделируемых молекул. Современные компьютеры позволяют рассчитывать системы состоящие из нескольких тысяч молекул.

Несмотря на то, что реальные системы состоят из порядка $10^{23} - 10^{30}$ молекул, молекулярно статистические методы позволяют получить как свойства сплошной фазы (например давление, химический потенциал, коэффициенты диффузии и т.д.), так и межфазные свойства (например поверхностное натяжение, адсорбцию на поверхности или в порах и т.д.).



\end{document}
