\chapter{Содержание}

\section*{Введение}

Молекулярно-статистические методы находят широкое применение в научных исследованиях. С развитием вычислительной техники сложность моделируемых систем возрастала. Усложнялись описание потенциального взаимодействия молекул, что позволило увеличить согласование расчета с экспериментом. Также увеличивалось количество моделируемых молекул. Современные компьютеры позволяют рассчитывать системы состоящие из нескольких тысяч молекул.

Несмотря на то, что реальные системы состоят из порядка $10^{23} - 10^{30}$ молекул, молекулярно статистические методы позволяют получить как свойства сплошной фазы (например давление, химический потенциал, коэффициенты диффузии и т.д.), так и межфазные свойства (например поверхностное натяжение, адсорбцию на поверхности или в порах и т.д.).

